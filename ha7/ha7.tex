\documentclass[fleqn]{article}
\usepackage[russian]{babel}
\usepackage[left=1in, right=1in, top=1in, bottom=1in]{geometry}
\usepackage{cmap}
\usepackage[utf8]{inputenc}
\usepackage{cmap}
\usepackage{amsfonts}
\usepackage{amssymb}
\usepackage[all]{xy}

\usepackage{mathtext} % русские буквы в фомулах
\usepackage{amsmath,amsfonts,amssymb,amsthm,mathtools,amsthm} % AMS
\usepackage{icomma} % "Умная" запятая: $0,2$ --- число, $0, 2$ --- перечисление
\usepackage{graphicx}

\usepackage{euscript}	 % Шрифт Евклид
\usepackage{mathrsfs} % Красивый матшрифт
\usepackage{mathtools}

\begin{document}

\begin{enumerate}

\item Показать, что каждый из следующих языков не является контекстно-свободным. Во всех пунктах будем 
предполагать, что такое $n$ действительно существует, после чего предоставлять слово $w$ на котором утверждение не 
работает.
	\begin{enumerate}
		\item $\left\{ a^ib^jc^k   \ | \ i < j < k \right\}$
		
		Рассмотрим слово $a^n b^{n + 1} c^{n + 2}$.
		
		Одновременно в $v, y$ не могут входить все три буквы, поэтому будет достаточно рассмотреть лишь случаи, когда 
		они покрывают какую то одну, либо какие-то две буквы.
		
		\begin{itemize}
			\item Только буквы $a$, при $k = 3$ получим количество $a$ большее чем букв $b$. Но такого быть не должно.
			\item Если только буквы $b$ или только $c$, то выберем $k = 0$, тогда $c$ будет столько же или меньше чем 
			$b$, а $b$ будет столько же или меньше чем $a$, такого тоже быть не может.
			\item В $v, y$ попали $a, b$. Тогда $k = 1$, количество $b$ столько же или больше чем $c$, чего быть не должно.
			\item В $v, y$ попали $b, c$. Тогда $k = 0$, количество $c$ столько же или меньше чем $a$, чего быть не должно.
		\end{itemize}
		\item $\left\{ a^nb^nc^i   \ | \ i \leq n \right\}$
		
		Рассмотрим слово $a^n b^n c^n$.
		Аналогично предыдущему, одновременно все три буквы войти не могут, разберём несколько случаев.
		\begin{itemize}
			\item Только $c$. Тогда выберем $k = 10$, получим $i > n$, этого быть не может.
			\item Случаи только $a$, только $b$, и $a, b$ в разных количествах слишком скучны, и ломаются при $k = 0$.
			\item Если в $v, y$ попало одинаковое число $a$ и $b$, тогда при $k = 0$ всё сломается, т.к. $c^n$ это 
			слишком много (мы уменьшили максимум для $i$).
		\end{itemize}
		
		
		\item $\left\{ 0^p         \ | \ p - \mathbf{prime} \right\}$
		
		Рассмотрим слово $0^{\hat{p}}$, где $\hat{p}$ - произвольное простое число, большее $n$ (оно существует, т.к. 
		простых чисел бесконечное количество). Накачивая это слово мы будем получать снова слово из нулей, в котором 
		будет уже $\hat{p} + |vy|k$ нулей. 
		
		Очевидно, что для всех значений $k$ мы не можем получать простые числа 
		(т.к. между простыми числами существует сколько угодно большие интервалы), значит при некотором $k_0$ значение 
		$\hat{p} + k_0|vy|$ будет составным, а значит и языку оно принадлежать не может.
		\item $\left\{ 0^i 1^j     \ | \ j = i^2 \right\}$
		
		Рассмотрим слово $0^n 1^{n^2}$. Случаи, когда $v, y$ содержат только один из символов 0 или 1 рассматривать не 
		будем - $k = 0$, очевидно, их ломает. 
		
		Рассмотрим случай, когда $0 \in vy\land 1 \in vy$.
		
		При $k = 0$ количество нулей уменьшится хотя бы на 1, значит, количество единиц должно быть не более чем $(n 
		- 1)^ 2 = n ^ 2 - 2n + 1$. Но единиц при этом мы убрали не более чем $n - 1$, значит их осталось более чем 
		$n^2 - n + 1$. Заметим, что $n ^ 2 - 2n + 1 < n^2 - n + 1$, значит, единиц слишком много, и полученное слово 
		не принадлежит языку.
		\item $\left\{ a^n b^n c^i \ | \ n \leq i \leq 2n \right\}$
		
		Рассмотрим слово $a^n b^n c^{2n}$. Доказательство для этого случая ничем не отличается от пункта $(b)$. Ниже \textit{copy-paste}
		\begin{itemize}
			\item Только $c$. Тогда выберем $k = 10$, получим $i > 2n$, этого быть не может.
			\item Случаи только $a$, только $b$, и $a, b$ в разных количествах слишком скучны, и ломаются при $k = 0$.
			\item Если в $v, y$ попало одинаковое число $a$ и $b$, тогда при $k = 0$ всё сломается, т.к. $c^{2n}$ это 
			слишком много (мы уменьшили максимум для $i$).
		\end{itemize}
		
		\item $\left\{ ww^Rw       \ | \ w \in \{0, 1\}^* \right\}$
		
		Рассмотрим слово (разделители стоят для удобства) $ww^Rw = 11..10|011..1|11..10, \mathbf{where} \ w = 
		1^n0$,заметим, что последний 0 отстоит от двух других более чем на $n$ позиций, значит все 3 нуля попасть в 
		$vy$ не могут. В другой стороны в этом случае, слове $ww^Rw$ должно быть всегда не менее трёх нулей. Значит, 
		ни один из нулей слова не может попадать в $vy$, т.к. если он туда попадёт, то при $k = 0$ нулей останется 
		меньше 3, и полученное слово точно будет не из языка.
		
		Осталось рассмотреть случае тогда в $vy$ попали только единички. При $k = 0$ первое $w$ перестанет 
		заканчиваться на $0$, т.к. его длина уменьшится, а третье $w$ по-прежнему будет оканчиваться на $0$, получили, 
		что $w \neq w$, противоречие, значит слово не из языка.
	\end{enumerate}
	
\item Придумать грамматики типа 1 или типа 0, задающие языки из первого задания; для каждой грамматики привести 
вывод какой-нибудь нетривиальной цепочки из языка.

\end{enumerate}

\end{document}
